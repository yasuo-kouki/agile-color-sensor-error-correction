\section{まとめ}
本実験では,スパース化・CSR 形式・int8 量子化を組み合わせることで,
ニューラルネットワークのモデルサイズを最大で元の 1/100 程度まで削減しつつ,
MSE の劣化を小さく抑えられることを示した.
しかし,学習条件やデータセットは各パラメータ数で共通の設定とし,
個別にハイパーパラメータ探索を行っていないため,
MSE とモデルサイズの関係について十分に最適化された結果とはいえない.
さらに,本研究で検討した量子化は 8 ビット固定であり,
4 ビット以下のより高圧縮な方式や層ごとのスケール最適化までは扱えていない.

今後は,各パラメータ数に対する学習条件のチューニングや複数回学習による統計的評価を行い,
より厳密に精度と削減率の関係を分析する必要がある.
また,Arduino側の推論時間・メモリ使用量・消費電力の最適化を行うことが必要である.
さらに,低ビット量子化やハードウェアに最適化した演算実装を検討することで,
より大規模なモデルを動作させるための実用的な指針を得ることが今後の課題である.