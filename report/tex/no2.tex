\subsection{実装方法}
計画書と異なる実装とした場合は、その理由(単に「実装が難しかったから」ではなく、第三者を納得させられる論理的な説明とすること)。

実装方法(アルゴリズム)の詳細、ソースコードの説明を以下に示す。

% ------------------------
% スパース化
% ------------------------
\subsubsection{スパース化の実装}

スパース化(sparsification)とは,配列中の小さな値を0に置き換える操作であり,
データの疎性を高めることで計算効率やメモリ使用量を削減する手法である.
本研究では,ある閾値(threshold)以下の値を0に置き換える簡単な方式を採用した.
具体的な実装をリスト\ref{lst:sparsify}に示す.

\begin{lstlisting}[caption=スパース化の実装,label=lst:sparsify]
def sparsify_array(arr: np.ndarray, threshold: float = 0.01):
    arr_copy = arr.copy()
    arr_copy[np.abs(arr_copy) < threshold] = 0
    return arr_copy
\end{lstlisting}

まず入力配列 \texttt{arr} をコピーし,変更可能な配列 \texttt{arr\_copy} を作成する.
続いて,絶対値が閾値 \texttt{threshold} 未満の要素を \texttt{0} に置き換える.
最終的に,スパース化された配列を返す.

この実装により,微小な値を0にすることでデータの疎性を高め,後続の計算処理の効率化やメモリ削減を実現できる.
また,閾値を調整することで,疎性の度合いを柔軟に制御可能である.

さらに,関数 \texttt{sparsify\_array} の処理の流れを図\ref{fig:sparsify_flow}に示す.

\begin{figure}[h]
    \centering
    \includegraphics[width=0.45\columnwidth]{../img/tex_2/no1.png}
    \caption{スパース化のフローチャート}
    \label{fig:sparsify_flow}
\end{figure}


% ------------------------
% 量子化
% ------------------------
\subsubsection{量子化の実装}

量子化とは,実数値を有限の整数値範囲に写像する操作であり,
ここでは代表的な8ビット量子化(int8 quantization)を採用した.
実装には NumPy を用い,入力配列をスケーリングして
8ビット整数範囲 \([ -128, 127 ]\) に丸め込む方式を採用した.
スケーリング係数(scale)は,入力配列の絶対値最大値が 127 に対応するように自動計算する.
具体的な実装をリスト\ref{lst:quantize}に示す.

\begin{lstlisting}[caption=量子化の実装,label=lst:quantize]
def int8_quantize(arr: np.ndarray, scale=None):
    if scale is None:
        max_abs = float(np.max(np.abs(arr)))
        scale = max_abs / 127.0 if max_abs != 0 else 1.0
    q = np.round(arr / scale).astype(np.int32)
    q = np.clip(q, -128, 127).astype(np.int8)
    return q, scale
\end{lstlisting}

まず入力配列 \texttt{arr} の絶対値最大値 \texttt{max\_abs} を求め,
それを 127 で割ることでスケーリング係数 \texttt{scale} を算出する.
続いて,配列を \texttt{scale} で割り整数に丸めた後,
範囲外の値を \texttt{np.clip} により \([ -128, 127 ]\) に制限し,
最終的に8ビット整数型(\texttt{np.int8})として返す.
また,同時にスケーリング係数を返すことで,後段の復元処理(dequantization)にも利用可能である.

さらに,関数 \texttt{int8\_quantize} の処理の流れを図\ref{fig:quantize_flow}に示す.

\begin{figure}[h]
    \centering
    \includegraphics[width=0.45\columnwidth]{../img/tex_2/no2.png}
    \caption{量子化のフローチャート}
    \label{fig:quantize_flow}
\end{figure}


% ------------------------
% CSR
% ------------------------
\subsubsection{CSR配列生成の実装}

本研究では,疎行列を効率的に格納するため,Compressed Sparse Row(CSR)形式への変換を行う関数 \texttt{to\_csr} を実装した.
CSR形式は,非ゼロ要素のみをデータ配列に保持し,行インデックスや列インデックスを別途管理することで,メモリ効率の高い行列表現を可能にする方式である.
本関数では入力テンソルを NumPy 配列に変換した後,量子化を行い,さらにCSR形式へ変換し,
C言語で利用可能な配列文字列として出力する.
具体的な実装をリスト\ref{lst:to_csr}に示す.

\begin{lstlisting}[caption=CSR配列生成の実装,label=lst:to_csr]
def to_csr(tensor: torch.Tensor, name: str, threshold: float = 0.01):
    arr = tensor.detach().numpy()
    arr_quantized, scale = quantize_array(arr, threshold)
    csr = csr_matrix(arr_quantized)
    data_str = f"int8_t {name}_data[] = {{ " + ", ".join(map(str, csr.data)) + " };"
    indices_str = f"int {name}_indices[] = {{ " + ", ".join(map(str, csr.indices)) + " };"
    indptr_str = f"int {name}_indptr[] = {{ " + ", ".join(map(str, csr.indptr)) + " };"
    shape_str = f"int {name}_shape[2] = {{{csr.shape[0]}, {csr.shape[1]}}};"
    scale_str = f"float {name}_scale = {scale};"
    return "\n".join([data_str, indices_str, indptr_str, shape_str, scale_str])
\end{lstlisting}

関数 \texttt{to\_csr} は以下の処理手順で動作する.
本実装により,量子化と疎行列化を組み合わせつつ,
C言語で直接利用可能な配列形式としてエクスポートすることが可能になる.
処理の概要を図\ref{fig:csr_flow}に示す.

\begin{figure}[h]
    \centering
    \includegraphics[width=0.9\columnwidth]{../img/tex_2/no3.jpg}
    \caption{CSR形式生成のフローチャート}
    \label{fig:csr_flow}
\end{figure}
